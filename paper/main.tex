\documentclass[UTF8,a4paper,11pt]{ctexart}

\usepackage{geometry}
\geometry{left=2.4cm,right=2.4cm,top=2.6cm,bottom=2.6cm}
\usepackage{amsmath,amssymb}
\usepackage{booktabs}
\usepackage{graphicx}
\usepackage{subcaption}
\usepackage{siunitx}
\usepackage{hyperref}
\hypersetup{colorlinks=true,linkcolor=blue,citecolor=blue,urlcolor=blue}

\title{近临界集群在追逃任务中的表现边界:\\基于任务内统计判定的大规模扫描研究}
\author{ChaseCrit Project}
\date{2026-02-06}

\begin{document}
\maketitle

\begin{abstract}
本文研究二维连续追逃场景中,逃跑者集群“更接近临界”是否能带来任务性能收益。与将无外场相变点直接迁移到任务场景不同,本文采用任务内统计判定:在同一任务设置内,以极化涨落易感性代理
\(
\chi = N_e \cdot \mathrm{Var}_t(P(t))
\)
(并辅以 \(\chi_{\text{local}}\)、\(\tau_{P,\mathrm{AR1}}\)、\(\xi_{\text{fluct}}\))比较“相对更临界”的参数区。实验基于容量受限、缓慢移动、多安全区机制,开展 \(w_{\text{align}}\) 扫描与追捕压力对比。主结果显示:在中等压力区间(\(v_p/v_e \approx 1.0\sim1.3\))内,生存率与 \(\chi\) 呈稳定正相关;在高压力(\(v_p/v_e=1.4\))下该关系显著减弱甚至失效。结论表明近临界优势不是全局性质,而是受任务外场强度调制的条件性现象。
\end{abstract}

\section{引言}
群体系统在相变附近常表现出高易感性与长程相关,理论上可能改善集体决策与扰动响应。追逃任务中,逃跑者既要保持协同,也要避免被追捕者利用可预测性。由此出现核心问题:临界性是否在对抗任务中稳定提升逃跑表现,还是仅在部分外场强度下有效。

本工作聚焦逃跑者集群,采用不可显式通信、仅局部可观测的设置,重点验证:
\begin{enumerate}
\item 任务内部更“临界”的参数区是否对应更高生存率;
\item 该关系是否随追捕压力变化而改变;
\item 无外场相变识别结论是否可直接迁移到任务最优。
\end{enumerate}

\section{任务与方法}
\subsection{场景机制}
采用二维连续空间,逃跑者不可显式通信,仅通过局部观测邻居与追捕者运动。安全区为多目标、容量受限、缓慢移动(G0)并随机刷新;当容量耗尽时立即失效并从逻辑上消失。默认捕获规则为瞬时捕获。

\subsection{策略参数}
核心扫描参数为
\(
w_{\text{align}} \in [0,1]
\),用于控制“跟随邻居行为”的强度。\(w_{\text{align}}\) 小表示更强从众,\(w_{\text{align}}\) 大表示更自主。

\subsection{任务内临界性判定}
本文不将“无外场 phase sweep 的峰值点”直接作为任务临界先验,而在同一任务设置内以统计量相对大小定义“更接近临界”。

主指标:
\begin{equation}
\chi = N_e \cdot \mathrm{Var}_t(P(t)),
\end{equation}
其中 \(P(t)\) 为存活逃跑者速度极化度。

辅指标包括:
\begin{itemize}
\item \(\chi_{\text{local}} = N_e \cdot \mathrm{Var}(P_{\text{local}})\);
\item \(\tau_{P,\mathrm{AR1}} = \frac{1+\rho_1}{1-\rho_1}\);
\item \(\xi_{\text{fluct}}\):速度涨落空间相关长度代理。
\end{itemize}

\subsection{实验与统计}
主要结果来自以下两组:
\begin{itemize}
\item 主扫描:\(v_p/v_e \in \{1.0,1.05,1.1,1.15,1.2\}\),\(w_{\text{align}}=0,0.05,\dots,1.0\),每格点 seeds=200;
\item 压力对比:\(v_p/v_e \in \{0.9,1.3,1.4\}\),其中 \(1.4\) 扩展至 seeds=240。
\end{itemize}
统计采用格点均值与 95\% CI,并报告跨 \(w_{\text{align}}\) 的 Pearson 相关系数 \(\mathrm{corr}(\mathrm{safe},\chi)\) 等。

\section{实验结果}
\subsection{主扫描结果(seeds=200)}
图~\ref{fig:main_safe_chi} 显示在 \(v_p/v_e=1.0\sim1.2\) 范围,\(w_{\text{align}}\) 对生存率与 \(\chi\) 具有共同结构:二者并非单调,且在中间区间出现较优表现带。

\begin{figure}[htbp]
\centering
\begin{subfigure}[b]{0.48\textwidth}
\includegraphics[width=\textwidth]{../doc/results_20260206_walign_task_internal_200seeds/figs/safe_vs_w_align.png}
\caption{\(\mathrm{safe\_frac}\) vs \(w_{\text{align}}\)}
\end{subfigure}
\hfill
\begin{subfigure}[b]{0.48\textwidth}
\includegraphics[width=\textwidth]{../doc/results_20260206_walign_task_internal_200seeds/figs/chi_vs_w_align.png}
\caption{\(\chi\) vs \(w_{\text{align}}\)}
\end{subfigure}
\caption{主扫描(\(v_p/v_e=1.0\sim1.2\), seeds=200)}
\label{fig:main_safe_chi}
\end{figure}

\subsection{压力对比与关系边界}
表~\ref{tab:corr} 总结不同追捕压力下的关键统计。可见:
\begin{itemize}
\item 中等压力区(\(1.0\sim1.3\))多呈 \(\mathrm{corr}(\mathrm{safe},\chi)>0\);
\item 高压力 \(v_p/v_e=1.4\) 时,\(\mathrm{corr}(\mathrm{safe},\chi)\approx 0\),且 \(\mathrm{corr}(\mathrm{safe},\xi_{\text{fluct}})<0\)。
\end{itemize}

\begin{table}[htbp]
\centering
\caption{任务内临界性与性能关系(按 \(w_{\text{align}}\) 格点均值)}
\label{tab:corr}
\begin{tabular}{cccccc}
\toprule
\(v_p/v_e\) & seeds & \(\mathrm{corr}(\mathrm{safe},\chi)\) & \(\mathrm{corr}(\mathrm{safe},\tau)\) & \(\mathrm{corr}(\mathrm{safe},\xi)\) & \(|w_{\mathrm{safe}}-w_{\chi}|\) \\
\midrule
0.90 & 120 & 0.272 & 0.272 & 0.236 & 0.05 \\
1.00 & 200 & 0.376 & 0.383 & 0.374 & 0.15 \\
1.05 & 200 & 0.441 & 0.369 & 0.371 & 0.35 \\
1.10 & 200 & 0.467 & 0.447 & 0.525 & 0.50 \\
1.15 & 200 & 0.621 & 0.585 & 0.414 & 0.30 \\
1.20 & 200 & 0.492 & 0.379 & 0.523 & 0.10 \\
1.30 & 120 & 0.378 & 0.408 & 0.320 & 0.45 \\
1.40 & 240 & 0.024 & 0.126 & -0.322 & 0.60 \\
\bottomrule
\end{tabular}
\end{table}

图~\ref{fig:pressure_scatter} 给出压力对比实验中的 \(\mathrm{safe}\)-\(\chi\) 散点,可见在高压力下耦合关系明显弱化。

\begin{figure}[htbp]
\centering
\includegraphics[width=0.68\textwidth]{../doc/results_20260206_walign_pressure_091314_sr14_240seeds/figs/scatter_safe_vs_chi.png}
\caption{压力对比实验中的 \(\mathrm{safe}\) 与 \(\chi\)(组均值散点)}
\label{fig:pressure_scatter}
\end{figure}

\section{讨论}
\subsection{为何“近临界优势”呈压力依赖}
在当前机制下,\(v_p/v_e\) 提升会增强外场约束。中等压力时,较高涨落与相关时间可能提升“群体重构+目标切换”的适应性;极高压力时,追捕过程主导系统动力学,内部涨落优势难以转化为生存收益。

\subsection{与噪声路线的关系}
无外场相变识别与任务内最优并非同一问题。本文结果支持以下分解:
\begin{itemize}
\item “任务内更临界(统计意义)是否更好”在部分压力区成立;
\item “外部 phase 点能否直接迁移为任务最优”当前不成立。
\end{itemize}

\subsection{局限}
当前追捕策略仍以近邻追逐为主,尚未充分刻画预测拦截与协作追捕;此外,障碍与反射边界下的稳健性验证仍不足。

\section{结论与下一步}
本文给出一个可复现结论:近临界与性能提升的关系不是全局真命题,而是受追捕压力调制的条件性关系。在 \(v_p/v_e=1.0\sim1.3\) 区间可观察到正耦合;在 \(v_p/v_e=1.4\) 时关系衰减。下一步将细化压力阈值(\(1.3\sim1.5\))、引入预测拦截追捕策略,并在反射边界/障碍场景复验。

\section*{复现信息}
主数据与报告:
\begin{itemize}
\item \texttt{doc/results\_20260206\_walign\_task\_internal\_200seeds}
\item \texttt{doc/results\_20260206\_walign\_pressure\_091314\_sr14\_240seeds}
\item \texttt{doc/论文准备-结果索引与复现清单.md}
\end{itemize}

\begin{thebibliography}{99}
\bibitem{vicsek1995}
Vicsek T, Czirók A, Ben-Jacob E, Cohen I, Shochet O.
Novel Type of Phase Transition in a System of Self-Driven Particles.
\textit{Physical Review Letters}, 1995, 75(6): 1226--1229.

\bibitem{chate2008}
Chaté H, Ginelli F, Grégoire G, Raynaud F.
Collective motion of self-propelled particles interacting without cohesion.
\textit{Physical Review E}, 2008, 77: 046113.

\bibitem{mora2011}
Mora T, Bialek W.
Are Biological Systems Poised at Criticality?
\textit{Journal of Statistical Physics}, 2011, 144: 268--302.

\bibitem{sumpter2010}
Sumpter D J T.
\textit{Collective Animal Behavior}.
Princeton University Press, 2010.
\end{thebibliography}

\end{document}
