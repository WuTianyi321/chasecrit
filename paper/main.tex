\documentclass[UTF8,a4paper,11pt]{ctexart}

\usepackage{geometry}
\geometry{left=2.4cm,right=2.4cm,top=2.6cm,bottom=2.6cm}
\usepackage{amsmath,amssymb}
\usepackage{booktabs}
\usepackage{graphicx}
\usepackage{subcaption}
\usepackage{siunitx}
\usepackage{longtable}
\usepackage{hyperref}
\hypersetup{colorlinks=true,linkcolor=blue,citecolor=blue,urlcolor=blue}

\title{追逃任务中的近临界优势:\\基于任务内统计判定的全量实验综合}
\author{ChaseCrit Project}
\date{2026-02-06}

\begin{document}
\maketitle

\begin{abstract}

Collective motion near critical points exhibits high susceptibility and long-range correlations, suggesting potential functional benefits for biological and artificial swarms in dynamic environments. However, whether such near-critical regimes enhance performance in adversarial pursuit-evasion tasks remains unclear. This study investigates whether evader swarms operating near critical collective states achieve higher survival rates in two-dimensional continuous pursuit-evasion scenarios with multiple capacity-limited safe zones.

We adopt a task-internal criterion for criticality, defining ``nearer-critical'' regions through statistical proxies including susceptibility $\chi = N_e \cdot \mathrm{Var}_t(P(t))$, local susceptibility $\chi_{\text{local}}$, correlation time $\tau$, and correlation length $\xi$. Through extensive parameter sweeps spanning alignment strength and pursuit intensity $v_p/v_e \in [0.9, 1.4]$ with $100$--$240$ random seeds per condition, we find that the relationship between criticality proxies and survival rates is strongly task-dependent.

Under moderate pursuit pressure ($v_p/v_e \approx 1.0\sim1.3$), survival rates correlate positively with criticality proxies ($r \approx 0.38$--$0.62$), with optimal performance at intermediate alignment strengths. However, under high pressure ($v_p/v_e = 1.4$), this relationship collapses ($r \approx 0.02$). Notably, critical points identified in phase-only settings (without pursuers or safe zones) do not transfer to task-optimal parameters, demonstrating that external forcing fundamentally alters the relevant critical regime.

These results establish that near-critical advantages in adversarial tasks are conditional rather than universal, depending critically on pursuit intensity. The findings suggest that effective swarm strategies require context-dependent tuning rather than fixed critical-point operation.

\end{abstract}

\textbf{Keywords:} collective motion, criticality, pursuit-evasion, active matter, swarm intelligence, phase transitions

\section{Introduction}

\subsection{Criticality and Collective Motion}

Collective motion in active matter systems exhibits rich phase behaviors that have been extensively studied in statistical physics~\cite{vicsek1995,chate2008,aldana2009}. The canonical Vicsek model demonstrates an order-disorder transition characterized by the emergence of long-range orientational order as noise decreases or density increases~\cite{vicsek1995,gregoire2004}. Near the critical point, these systems display hallmark features including diverging susceptibility, power-law correlations, and enhanced response to perturbations~\cite{toner1995,aldana2009}.

Such critical phenomena have inspired speculation about functional consequences in biological systems. The ``criticality hypothesis'' suggests that biological networks may self-organize near critical points to optimize information processing, sensitivity to environmental changes, and adaptive response~\cite{mora2011,bialek2012}. In collective animal behavior, starling flocks have been reported to exhibit scale-free correlations reminiscent of critical systems~\cite{bialek2012}, though the interpretation remains debated~\cite{chaigneau2022}.

\subsection{The Pursuit-Evasion Context}

While criticality has been studied extensively in equilibrium and non-equilibrium systems, its relevance to adversarial scenarios remains poorly understood. Pursuit-evasion represents a fundamental class of adversarial interactions ubiquitous in biological and engineered systems---from predator-prey dynamics to autonomous drone surveillance~\cite{sumpter2010,isaac2011,vasarhelyi2018}.

In such tasks, evaders face a fundamental trade-off: cohesion enables collective defense and information sharing, but excessive order creates predictability that pursuers can exploit~\cite{strobl2022,hsieh2022}. This tension suggests that optimal evasion strategies may involve intermediate regimes balancing order and disorder. Whether such regimes correspond to critical states of collective motion, and whether criticality genuinely enhances performance, remains an open question.

\subsection{Challenges in Defining Criticality for Adversarial Tasks}

A fundamental challenge in studying criticality in task contexts is defining what ``critical'' means when external forcing is present. Traditional phase identification removes external fields to locate intrinsic transition points. However, in pursuit-evasion, the ``external field'' (pursuers) is inseparable from the task itself.

This creates two distinct questions that are often conflated:
\begin{enumerate}
    \item Does the collective state that performs best in the task correspond to the intrinsic critical point of the collective dynamics?
    \item Within the task itself, do states with higher criticality proxies (susceptibility, correlation length) achieve better performance?
\end{enumerate}

Previous work has not systematically distinguished these questions. Most studies either examine collective dynamics without adversarial forcing~\cite{vicsek1995,chate2008} or study pursuit-evasion without reference to collective phase behavior~\cite{hsieh2022,vasarhelyi2018}. The intersection---whether and when critical collective states benefit adversarial performance---remains largely unexplored.

\subsection{Present Study}

This study addresses these gaps through systematic investigation of pursuit-evasion with evader swarms parameterized to span ordered, critical, and disordered collective regimes. We employ a task-internal criterion for criticality: rather than assuming external phase points transfer to the task, we measure criticality proxies (susceptibility, correlation length, correlation time) within each task setting and examine their relationship with survival performance.

Our experimental design spans two control parameter routes: alignment strength $w_{\text{align}}$ and angular noise $\eta$. For each route, we conduct high-sample sweeps ($n=100$--$240$ seeds per condition) across pursuit intensities $v_p/v_e \in [0.9, 1.4]$, enabling robust statistical characterization of the criticality-performance relationship and its dependence on adversarial pressure.

The results reveal that near-critical advantages are conditional rather than universal. Under moderate pursuit pressure, survival rates correlate positively with criticality proxies, consistent with the intuition that enhanced susceptibility aids collective response. However, this relationship weakens and eventually collapses as pursuit intensity increases, suggesting that optimal strategies shift toward predictability minimization under severe threat.

\subsection{Contributions}

The main contributions of this work are:
\begin{enumerate}
    \item A systematic framework for evaluating criticality in adversarial task settings using task-internal statistical proxies.
    \item Empirical evidence that the criticality-performance relationship is task-dependent, holding under moderate but not high pursuit pressure.
    \item Demonstration that intrinsic phase critical points (without forcing) do not transfer to task-optimal parameters when external forcing is present.
    \item Identification of intermediate alignment regimes as generally optimal for evasion, with the specific optimum shifting toward lower alignment as pressure increases.
\end{enumerate}

These findings advance understanding of collective behavior in adversarial contexts and provide guidance for designing adaptive swarm strategies that tune collective organization to environmental pressure.

\section{任务与方法}
\subsection{场景、状态变量与边界映射}
环境为二维连续空间 \(\Omega=[0,L_x)\times[0,L_y)\),离散时间步长为 \(\Delta t\)。
逃跑者集合记为 \(\mathcal{E}\)(规模 \(N_e\)),追捕者集合记为 \(\mathcal{P}\)(规模 \(N_p\)),安全区集合记为 \(\mathcal{Z}\)(动态规模,最多 \(K_{\max}\) 个有效区)。
状态变量包括:
\begin{itemize}
\item 逃跑者位置/速度 \((x_i(t),v_i(t))\), \(i\in\mathcal{E}\);
\item 追捕者位置/速度 \((y_m(t),u_m(t))\), \(m\in\mathcal{P}\);
\item 安全区位置/速度 \((z_k(t),q_k(t))\)、容量 \(C_k\)、占用 \(O_k(t)\)、有效标记 \(a_k(t)\)。
\end{itemize}
边界映射记为 \(\mathcal{B}(\cdot)\):
\begin{itemize}
\item 周期边界:\(\mathcal{B}(x)=x\bmod (L_x,L_y)\);
\item 反射边界:按 \(x^\ast=\mathrm{mod}(x,2L)\) 折叠到 \([0,L]\),若发生折返则对应速度分量取反。
\end{itemize}
仿真中每一步先更新安全区,再更新逃跑者与追捕者,最后执行捕获与入区判定。

\subsection{逃跑者运动更新}
每步仅对仍存活且未入区个体(记作 \(\mathcal{A}(t)\))更新。对 \(i\in\mathcal{A}(t)\),定义周期最短位移
\(\Delta_{ij}(t)\)、\(\Delta_{im}^{(p)}(t)\)、\(\Delta_{ik}^{(z)}(t)\)。
单位化算子记为 \(\mathrm{unit}(\cdot)\)。

邻域对齐与排斥项:
\begin{align}
d_i^{\text{align}} &= \mathrm{unit}\!\left(\sum_{j\neq i,\ \|\Delta_{ij}\|^2\le r_{\text{nbr}}^2}\mathrm{unit}(v_j)\right),\\
d_i^{\text{sep}} &= s_{\text{sep}}\cdot \mathrm{unit}\!\left(-\sum_{j\neq i,\ \|\Delta_{ij}\|^2\le r_{\text{sep}}^2}\frac{\Delta_{ij}}{\max(\|\Delta_{ij}\|^2,\varepsilon)}\right).
\end{align}
追捕规避项:
\begin{equation}
d_i^{\text{avoid}}=\mathrm{unit}\!\left(-\sum_{m,\ \|\Delta_{im}^{(p)}\|^2\le r_{\text{pred}}^2}\frac{\Delta_{im}^{(p)}}{\max(\|\Delta_{im}^{(p)}\|^2,\varepsilon)}\right).
\end{equation}
目标项:在可检测有效安全区集合
\(\mathcal{Z}_i(t)=\{k:a_k(t)=1,\ \|\Delta_{ik}^{(z)}\|^2\le r_{\text{detect}}^2\}\)
中选最近区 \(k^\star\)。若 \(\mathcal{Z}_i(t)\neq\varnothing\),
\begin{equation}
d_i^{\text{goal}}=\mathrm{unit}(\Delta_{ik^\star}^{(z)}),\quad I_i^{\text{goal}}=1;
\end{equation}
否则 \(I_i^{\text{goal}}=0\),探索方向为当前航向(若速度近零则随机单位向量)\(d_i^{\text{explore}}\)。

合成方向采用两种控制模式:
\begin{itemize}
\item 传统独立权重模式(legacy):
\begin{equation}
\tilde d_i = w_{\text{align}}d_i^{\text{align}}
+ w_{\text{avoid}}d_i^{\text{avoid}}
+ w_{\text{goal}}I_i^{\text{goal}}d_i^{\text{goal}}
+ w_{\text{explore}}(1-I_i^{\text{goal}})d_i^{\text{explore}}
+ d_i^{\text{sep}}.
\end{equation}
\item 单参数占比模式(share):令 \(\lambda\in[0,1]\) 表示对齐项占比(实现中 \(\lambda=w_{\text{align}}\)),先构造非对齐复合方向
\begin{equation}
d_i^{\text{non}}=\mathrm{unit}\!\left(
w_{\text{avoid}}d_i^{\text{avoid}}
+ w_{\text{goal}}I_i^{\text{goal}}d_i^{\text{goal}}
+ w_{\text{explore}}(1-I_i^{\text{goal}})d_i^{\text{explore}}
+ d_i^{\text{sep}}
\right),
\end{equation}
再按
\begin{equation}
\tilde d_i=\lambda d_i^{\text{align}}+(1-\lambda)d_i^{\text{non}}.
\end{equation}
\end{itemize}
令 \(d_i=\mathrm{unit}(\tilde d_i)\),并施加角噪声
\(\theta_i\sim\mathcal U[-\eta,\eta]\)(\(\eta=\text{angle\_noise}\)):
\begin{equation}
d_i^{\eta}=\mathrm{unit}(R(\theta_i)d_i).
\end{equation}
速度采用惯性混合更新:
\begin{equation}
v_i(t+1)=(1-\alpha)v_i(t)+\alpha v_e d_i^{\eta},\quad \alpha=\text{inertia}.
\end{equation}
位置更新为
\begin{equation}
x_i(t+1)=\mathcal{B}\!\left(x_i(t)+v_i(t+1)\Delta t\right).
\end{equation}

\subsection{追捕者更新、捕获与入区规则}
追捕者速度上限设为 \(v_p=\gamma v_e\)(\(\gamma=v_p/v_e\))。
当前策略为最近邻追逐:
\begin{equation}
j^\star(m)=\arg\min_{j\in\mathcal{A}(t)}\|\Delta_{mj}^{(e)}(t)\|^2,\qquad
u_m(t+1)=v_p\cdot \mathrm{unit}\!\left(\Delta_{mj^\star(m)}^{(e)}(t)\right),
\end{equation}
\begin{equation}
y_m(t+1)=\mathcal{B}\!\left(y_m(t)+u_m(t+1)\Delta t\right).
\end{equation}

捕获规则为瞬时捕获:若存在 \(m\) 使 \(\|x_i-y_m\|\le r_{\text{cap}}\),则逃跑者 \(i\) 在该步记为 captured。

入区规则按安全区逐个处理:对有效区 \(k\),找出 \(\|x_i-z_k\|\le r_{\text{safe}}\) 的候选并随机打乱;按顺序填充容量,若
\(O_k(t)\ge C_k\) 则该区立即失效(逻辑消失)。

\subsection{安全区 G0 缓慢移动与刷新机制}
安全区采用 G0 常速随机游走。每隔 \(T_{\text{turn}}\) 步,方向旋转角
\(\phi\sim\mathcal U[-\phi_{\max},\phi_{\max}]\),随后归一化并乘以常速 \(v_z\);位置更新:
\begin{equation}
z_k(t+1)=\mathcal{B}\!\left(z_k(t)+q_k(t+1)\Delta t\right).
\end{equation}
刷新规则:
\begin{itemize}
\item 若当前有效区数为 0,则强制刷新 1 个安全区;
\item 若有效区数 \(<K_{\max}\),以概率 \(p_{\text{spawn}}\) 触发随机刷新;
\item 反射边界下刷新点从边界采样,周期边界下从域内均匀采样;
\item 刷新点需满足与其他安全区、追捕者(及障碍,如启用)的最小距离约束。
\end{itemize}

\subsection{参数固定项与扫描项(写明 \(w_{\text{align}}\) 扫描时其余参数)}
表~\ref{tab:param_protocol} 给出本文主结论所用实验族的固定参数与扫描参数。除特别注明外,未扫描参数均固定为表中数值。

\begin{table}[htbp]
\centering
\caption{主要实验族的参数协议(固定项与扫描项)}
\label{tab:param_protocol}
\begin{tabular}{p{2.1cm}p{5.8cm}p{5.0cm}}
\toprule
实验族 & 固定参数(核心) & 扫描参数 \\
\midrule
E07(任务内 \(w_{\text{align}}\), 200 seeds) &
\(N_e=128\), \(N_p=2\), \(v_e=1.0\), inertia \(=0.35\), \(r_{\text{nbr}}=6.0\), \(r_{\text{pred}}=10.0\), \(r_{\text{detect}}=4.0\), \(r_{\text{sep}}=1.25\), \(s_{\text{sep}}=1.0\), \(w_{\text{goal}}=1.2\), \(w_{\text{avoid}}=1.5\), \(w_{\text{explore}}=0.4\), angle\_noise \(=0.12\), boundary=periodic, \(K_{\max}=4\), \(r_{\text{safe}}=1.0\), \(v_z=0.1\), steps \(=600\) &
\(w_{\text{align}}\in\{0,0.05,\ldots,1.0\}\),
\(v_p/v_e\in\{1.0,1.05,1.1,1.15,1.2\}\) \\
\midrule
E09(压力对比) &
与 E07 相同(含 angle\_noise \(=0.12\)) &
\(w_{\text{align}}\in\{0,0.05,\ldots,1.0\}\),
\(v_p/v_e\in\{0.9,1.3,1.4\}\) \\
\midrule
E10/E11(任务噪声) &
与 E07 相同,且 \(v_p/v_e=1.1\);E10 固定 \(w_{\text{align}}=0.6\),E11 固定 \(w_{\text{align}}=1.0\) &
angle\_noise \(\in\{0,0.2,\ldots,3.0\}\) \\
\midrule
E12(phase 噪声识别) &
\(N_p=0\), \(K_{\max}=0\), \(w_{\text{goal}}=w_{\text{avoid}}=w_{\text{explore}}=0\), \(w_{\text{align}}=1.0\), 其余动力学参数同上, steps \(=1200\) &
angle\_noise \(\in\{0,0.2,\ldots,3.0\}\) \\
\bottomrule
\end{tabular}
\end{table}

单参数占比路线(share)用于后续实验:固定 angle\_noise \(=0\),扫描 \(\lambda=w_{\text{align}}\in[0,1]\),其余权重 \(w_{\text{goal}},w_{\text{avoid}},w_{\text{explore}}\) 固定,从而避免“多权重同时竞争”带来的解释歧义。

\subsection{任务内临界性判定}
主判定指标为
\begin{equation}
\chi = N_e \cdot \mathrm{Var}_t(P(t)),
\end{equation}
其中 \(P(t)\) 为存活逃跑者速度极化度。辅助指标为
\(\chi_{\text{local}}=N_e\cdot \mathrm{Var}(P_{\text{local}})\)、
\(\tau_{P,\mathrm{AR1}}=\frac{1+\rho_1}{1-\rho_1}\)、
\(\xi_{\text{fluct}}\)(涨落相关长度代理)。
本文将“更近临界”定义为同一任务设置内这些统计量的相对升高,而非外部 phase 峰值的直接迁移。

\subsection{统计协议}
每个参数格点采用多随机种子重复,报告均值与 95\% CI。相关性分析使用格点均值层面的 Pearson(必要时补充 Spearman)系数。

\section{实验结果(全量批次综合)}
\subsection{批次总览}
为覆盖全部已完成实验,本文将有结构化聚合表(可计算统计量)的批次记为 E 系列,将仅用于烟雾测试或流程验证、未保留完整聚合表的批次记为 P 系列。

\begin{longtable}{p{1.2cm}p{3.4cm}p{1.7cm}p{1.7cm}p{1.9cm}p{3.5cm}}
\caption{E 系列(可量化)实验批次总览与关键统计}\label{tab:all_quant}\\
\toprule
ID & 实验类型 & seeds/格点 & steps & 最优 \(\mathrm{safe}\) & \(\mathrm{safe}\)-\(\chi\) 关系 \\
\midrule
\endfirsthead
\toprule
ID & 实验类型 & seeds/格点 & steps & 最优 \(\mathrm{safe}\) & \(\mathrm{safe}\)-\(\chi\) 关系 \\
\midrule
\endhead
E01 & 固定追捕者扫描(初版) & 10 & 600 & \(sr=1.00,w=0.30\), 0.3055 & 指标版本不含 \(\chi\) \\
E02 & 固定追捕者扫描(扩展) & 15 & 600 & \(sr=1.00,w=0.65\), 0.4214 & 平均相关 0.120 \\
E03 & 固定追捕者扫描(50 seeds) & 50 & 600 & \(sr=1.00,w=0.35\), 0.3966 & 平均相关 0.372 \\
E04 & 固定追捕者扫描(50 seeds, 指标增强) & 50 & 600 & \(sr=1.00,w=0.35\), 0.3966 & 平均相关 0.372 \\
E05 & 固定追捕者扫描(100 seeds, \(sr\in\{1.1,1.2\}\)) & 100 & 600 & \(sr=1.10,w=0.70\), 0.3725 & 平均相关 0.399 \\
E06 & 任务内 \(w_{\text{align}}\) 扫描(中样本) & 80 & 600 & \(sr=1.00,w=0.95\), 0.3900 & 平均相关 0.401 \\
E07 & 任务内 \(w_{\text{align}}\) 扫描(高样本) & 200 & 600 & \(sr=1.00,w=0.35\), 0.3893 & 平均相关 0.479 \\
E08 & 压力对比(0.9/1.3/1.4) & 120 & 600 & \(sr=0.90,w=0.55\), 0.4092 & 平均相关 0.193 \\
E09 & 压力对比(\(sr=1.4\)扩样) & 120/240 & 600 & \(sr=0.90,w=0.55\), 0.4092 & 平均相关 0.225 \\
E10 & 任务噪声扫描(\(w=0.6\)) & 100 & 600 & noise=0.00, 0.3567 & Pearson 0.814 \\
E11 & 任务噪声扫描(\(w=1.0\)) & 100 & 600 & noise=0.20, 0.3387 & Pearson 0.824 \\
E12 & Phase 噪声识别(无追捕/无安全区) & 100 & 1200 & \(\mathrm{safe}\) 不适用 & \(\chi\) 峰值在 noise=1.80 \\
\bottomrule
\end{longtable}

\begin{longtable}{p{1.2cm}p{4.2cm}p{2.0cm}p{6.3cm}}
\caption{P 系列(流程/烟雾)批次及其作用}\label{tab:all_pilot}\\
\toprule
ID & 批次类型 & 是否量化汇总 & 作用 \\
\midrule
\endfirsthead
\toprule
ID & 批次类型 & 是否量化汇总 & 作用 \\
\midrule
\endhead
P01 & Phase smoke & 否 & 验证 phase-sweep 管线与绘图流程可用。 \\
P02 & Phase baseline & 否 & 早期 phase 识别趋势观察。 \\
P03 & Phase wide & 否 & 扩大噪声范围后的早期可视化验证。 \\
P04 & Phase wider & 否(仅图) & 后续高样本 phase 扫描前的参数摸底。 \\
P05 & Task noise smoke & 否 & 验证任务噪声扫描管线与散点图生成。 \\
P06 & Task noise early (\(w=1.0\)) & 否 & 任务噪声路线早期趋势勘探。 \\
\bottomrule
\end{longtable}

\subsection{早期到高样本的演化结果(E01--E06)}
E01--E05 展示了固定追捕者数量扫描从低样本到高样本、从单指标到多指标的演化过程。E01 中尚未纳入 \(\chi\) 指标;E02 开始引入 \(\chi\),并首次在格点均值层面观察到正相关(平均相关 0.120);E03/E04 在 50 seeds 下将相关提升到 0.372;E05 在 100 seeds、\(v_p/v_e\in\{1.1,1.2\}\) 下维持正相关(0.399)。

E06 作为任务内 \(w_{\text{align}}\) 路线的中样本版本(80 seeds),与 E07(200 seeds)在趋势方向上保持一致:\(\mathrm{safe}\) 与 \(\chi\) 的整体耦合为正,但最优 \(w_{\text{align}}\) 呈带状而非尖峰点。

\begin{figure}[htbp]
\centering
\begin{subfigure}[b]{0.48\textwidth}
\includegraphics[width=\textwidth]{../doc/results_20260205_fixedNp_scan_expanded/figs/safe_vs_w_align.png}
\caption{E02: \(\mathrm{safe}\) vs \(w_{\text{align}}\)}
\end{subfigure}
\hfill
\begin{subfigure}[b]{0.48\textwidth}
\includegraphics[width=\textwidth]{../doc/results_20260205_fixedNp_scan_expanded/figs/chi_vs_w_align.png}
\caption{E02: \(\chi\) vs \(w_{\text{align}}\)}
\end{subfigure}
\caption{早期固定追捕者扫描中引入 \(\chi\) 后的趋势}
\label{fig:e02_early}
\end{figure}

\begin{figure}[htbp]
\centering
\begin{subfigure}[b]{0.48\textwidth}
\includegraphics[width=\textwidth]{../doc/results_20260206_fixedNp_sr11_sr12_100seeds/figs/safe_vs_w_align.png}
\caption{E05: \(\mathrm{safe}\) vs \(w_{\text{align}}\)}
\end{subfigure}
\hfill
\begin{subfigure}[b]{0.48\textwidth}
\includegraphics[width=\textwidth]{../doc/results_20260206_fixedNp_sr11_sr12_100seeds/figs/chi_vs_w_align.png}
\caption{E05: \(\chi\) vs \(w_{\text{align}}\)}
\end{subfigure}
\caption{高样本固定追捕者扫描下的主趋势延续}
\label{fig:e05_highsample}
\end{figure}

\subsection{\(w_{\text{align}}\) 主线与压力对比}
E07 与 E09 构成本文主结果骨架。E07(高样本)显示在 \(v_p/v_e=1.0\sim1.2\) 范围内,\(\mathrm{safe}\) 与 \(\chi\) 整体正相关;E09 显示该关系在高压力端(\(v_p/v_e=1.4\))显著减弱。

\begin{figure}[htbp]
\centering
\begin{subfigure}[b]{0.48\textwidth}
\includegraphics[width=\textwidth]{../doc/results_20260206_walign_task_internal_200seeds/figs/safe_vs_w_align.png}
\caption{E07: \(\mathrm{safe}\) vs \(w_{\text{align}}\)}
\end{subfigure}
\hfill
\begin{subfigure}[b]{0.48\textwidth}
\includegraphics[width=\textwidth]{../doc/results_20260206_walign_task_internal_200seeds/figs/chi_vs_w_align.png}
\caption{E07: \(\chi\) vs \(w_{\text{align}}\)}
\end{subfigure}
\caption{高样本任务内 \(w_{\text{align}}\) 扫描结果}
\label{fig:e07_main}
\end{figure}

\begin{figure}[htbp]
\centering
\includegraphics[width=0.68\textwidth]{../doc/results_20260206_walign_pressure_091314_sr14_240seeds/figs/scatter_safe_vs_chi.png}
\caption{E09: 压力对比中的 \(\mathrm{safe}\)-\(\chi\) 关系}
\label{fig:e09_pressure}
\end{figure}

\subsection{E09 离散度诊断}
在 E09 合并散点中,\(\chi\approx4.5\) 区域存在明显纵向离散。按压力分层后可见该离散主要来自 \(v_p/v_e\) 层间基线差异:相同 \(\chi\) 水平下,\(v_p/v_e=0.9\) 的 \(\mathrm{safe}\) 系统性高于 \(1.3\) 与 \(1.4\)。在 \(\chi\in[4.2,4.8]\) 区间,\(\mathrm{safe}\) 方差约 96.7\% 可由层间差异解释,说明该现象主要是条件混合效应而非单一指标噪声。

若不做任何聚合、直接在单次运行层面考察 \((\chi,\mathrm{safe\_frac})\),相关性较弱:E09 全部原始点的 Pearson 相关约为 \(-0.016\),且各压力层分别约为 \(-0.025\)(\(v_p/v_e=0.9\))、\(0.051\)(\(1.3\))与 \(0.018\)(\(1.4\))。因此,本文对“近临界优势”的讨论以参数格点统计量(多 seed 汇总)为主,并将原始点云解释为高随机性背景下的微观离散。

\begin{figure}[htbp]
\centering
\includegraphics[width=0.68\textwidth]{../doc/results_20260206_walign_pressure_091314_sr14_240seeds/figs/scatter_safe_vs_chi_by_sr.png}
\caption{E09: 按压力分层后的 \(\mathrm{safe}\)-\(\chi\) 散点}
\label{fig:e09_by_sr}
\end{figure}

\subsection{噪声路线全量结果(关键对照)}
E10 与 E11 均显示任务噪声扫描中 \(\mathrm{safe}\)-\(\chi\) 强正相关。尤其 E11(\(w_{\text{align}}=1.0\))给出 Pearson 0.824、Spearman 0.965。E12 则表明无外场 phase 设置中 \(\chi\) 峰值位于 noise=1.80,与任务最优噪声不一致。

\begin{table}[htbp]
\centering
\caption{噪声路线关键统计(修正后的峰值位置与数值)}
\label{tab:noise_key}
\begin{tabular}{cccccccc}
\toprule
ID & 设置 & seeds & noise@safe\(_{\max}\) & safe\(_{\max}\) & noise@\(\chi_{\max}\) & \(\chi_{\max}\) & \(\mathrm{corr}(\mathrm{safe},\chi)\) \\
\midrule
E10 & Task, \(w=0.6\) & 100 & 0.00 & 0.3567 & 0.00 & 4.2812 & 0.814 \\
E11 & Task, \(w=1.0\) & 100 & 0.20 & 0.3387 & 0.00 & 4.2378 & 0.824 \\
E12 & Phase & 100 & -- & -- & 1.80 & 13.2085 & -- \\
\bottomrule
\end{tabular}
\end{table}

\begin{figure}[htbp]
\centering
\begin{subfigure}[b]{0.48\textwidth}
\includegraphics[width=\textwidth]{../doc/results_20260206_task_noise_w10_sr11_100seeds/figs/safe_vs_noise.png}
\caption{E11: \(\mathrm{safe}\) vs noise}
\end{subfigure}
\hfill
\begin{subfigure}[b]{0.48\textwidth}
\includegraphics[width=\textwidth]{../doc/results_20260206_task_noise_w10_sr11_100seeds/figs/scatter_safe_vs_chi.png}
\caption{E11: \(\mathrm{safe}\)-\(\chi\) 散点}
\end{subfigure}
\caption{E11(\(w=1.0\))中观察到的强正相关}
\label{fig:e11_noise}
\end{figure}

\begin{figure}[htbp]
\centering
\begin{subfigure}[b]{0.48\textwidth}
\includegraphics[width=\textwidth]{../doc/results_20260206_task_noise_w06_sr11_100seeds/figs/scatter_safe_vs_chi.png}
\caption{E10: Task 噪声扫描对照}
\end{subfigure}
\hfill
\begin{subfigure}[b]{0.48\textwidth}
\includegraphics[width=\textwidth]{../doc/results_20260206_phase_noise_100seeds_steps1200/figs/chi_vs_noise.png}
\caption{E12: Phase 噪声识别}
\end{subfigure}
\caption{任务噪声与 phase 噪声结果的并列比较}
\label{fig:e10_e12}
\end{figure}

\subsection{全量批次图表索引}
为确保正文覆盖全部批次,表~\ref{tab:figure_index} 给出 E/P 系列对应的代表性图表类别。E 系列均对应完整趋势图与散点图;P 系列用于流程与绘图链路验证,其图表仅用于方法学完备性说明,不参与主要定量结论。

\begin{table}[htbp]
\centering
\caption{全量批次代表性图表索引(按实验编号)}
\label{tab:figure_index}
\begin{tabular}{ccc}
\toprule
实验编号 & 主要图表类别 & 在本文中的作用 \\
\midrule
E01--E05 & \(w_{\text{align}}\) 趋势图与热图 & 固定追捕者扫描基线与扩样 \\
E06--E07 & 任务内 \(w_{\text{align}}\) 趋势与散点 & 近临界主线证据(中样本/高样本) \\
E08--E09 & 压力分层趋势与散点 & 外场强度对耦合关系的调制 \\
E10--E11 & 噪声趋势与 \(\mathrm{safe}\)-\(\chi\) 散点 & 任务噪声路线与强正相关检验 \\
E12 & phase 噪声趋势图 & 任务判定与 phase 判定分离 \\
P01--P06 & 简化趋势图(烟雾/流程) & 管线连通性与早期参数摸底 \\
\bottomrule
\end{tabular}
\end{table}

\section{讨论}
\subsection{近临界优势的条件性}
E07/E09/E10/E11/E12 的综合结果支持一个一致结论:近临界优势并非在所有外场强度下成立。中等压力下,\(\chi\) 增大通常伴随更高生存率;高压力下(E09 的 \(v_p/v_e=1.4\))该耦合显著弱化。

\subsection{任务内判定与 phase 判定的分离}
E12 的峰值噪声(1.80)并未在 E10/E11 中转化为任务最优噪声。该现象说明“无外场相变识别”与“有外场任务最优”是两个不同层次的问题,应避免直接等同。

\subsection{关于“低噪声高 \(\chi\)”的解释一致性}
在 E10/E11 中,低噪声区域同时呈现更高 \(\chi\) 与更高生存率。若判定标准限定为任务内统计表现,则该区域可被解释为“更接近任务内近临界区”,并不与本文框架矛盾。本文强调的限制仅在于:该“任务内近临界”不应被误读为“无外场 phase 临界点已迁移为任务最优点”。

\subsection{跨批次一致性}
从 E01 到 E11,随着样本量上升和指标增强,核心趋势保持一致:中间参数带通常优于极端参数带;高涨落代理对性能的解释力随场景压力变化而变化。

\subsection{局限}
现有追捕策略主要是近邻追逐,尚未系统引入预测拦截与协作分工;此外,障碍与反射边界下的高样本复验仍需补齐。

\section{结论}
本文以任务内统计判定为核心,综合了全部已完成实验批次并给出统一结论:
\begin{enumerate}
\item 近临界与性能提升之间存在可重复的正耦合,但具有明确条件性;
\item 该耦合在中等追捕压力下更明显,在高压力下可能失效;
\item 无外场 phase 峰值不能直接迁移为任务最优参数。
\end{enumerate}

下一步工作包括:细化 \(v_p/v_e=1.3\sim1.5\) 阈值区间、引入预测拦截/协作追捕策略、并在反射边界与障碍场景进行高样本稳健性复验。


\section*{复现说明}
本文所有结论均来自统一代码库下的批处理实验。正文通过实验编号(E/P)给出完整批次覆盖,并在表~\ref{tab:all_quant} 与表~\ref{tab:all_pilot} 中明确各批次样本规模、目的与关键统计。读者可直接根据实验编号复核对应图表与统计结论。



\begin{thebibliography}{99}
\bibitem{vicsek1995}
Vicsek T, Czirók A, Ben-Jacob E, Cohen I, Shochet O.
Novel Type of Phase Transition in a System of Self-Driven Particles.
\textit{Physical Review Letters}, 1995, 75(6): 1226--1229.

\bibitem{chate2008}
Chaté H, Ginelli F, Grégoire G, Raynaud F.
Collective motion of self-propelled particles interacting without cohesion.
\textit{Physical Review E}, 2008, 77: 046113.

\bibitem{mora2011}
Mora T, Bialek W.
Are Biological Systems Poised at Criticality?
\textit{Journal of Statistical Physics}, 2011, 144: 268--302.

\bibitem{sumpter2010}
Sumpter D J T.
\textit{Collective Animal Behavior}.
Princeton University Press, 2010.
\end{thebibliography}

\end{document}
