\begin{abstract}

Collective motion near critical points exhibits high susceptibility and long-range correlations, suggesting potential functional benefits for biological and artificial swarms in dynamic environments. However, whether such near-critical regimes enhance performance in adversarial pursuit-evasion tasks remains unclear. This study investigates whether evader swarms operating near critical collective states achieve higher survival rates in two-dimensional continuous pursuit-evasion scenarios with multiple capacity-limited safe zones.

We adopt a task-internal criterion for criticality, defining ``nearer-critical'' regions through statistical proxies including susceptibility $\chi = N_e \cdot \mathrm{Var}_t(P(t))$, local susceptibility $\chi_{\text{local}}$, correlation time $\tau$, and correlation length $\xi$. Through extensive parameter sweeps spanning alignment strength and pursuit intensity $v_p/v_e \in [0.9, 1.4]$ with $100$--$240$ random seeds per condition, we find that the relationship between criticality proxies and survival rates is strongly task-dependent.

Under moderate pursuit pressure ($v_p/v_e \approx 1.0\sim1.3$), survival rates correlate positively with criticality proxies ($r \approx 0.38$--$0.62$), with optimal performance at intermediate alignment strengths. However, under high pressure ($v_p/v_e = 1.4$), this relationship collapses ($r \approx 0.02$). Notably, critical points identified in phase-only settings (without pursuers or safe zones) do not transfer to task-optimal parameters, demonstrating that external forcing fundamentally alters the relevant critical regime.

These results establish that near-critical advantages in adversarial tasks are conditional rather than universal, depending critically on pursuit intensity. The findings suggest that effective swarm strategies require context-dependent tuning rather than fixed critical-point operation.

\end{abstract}

\textbf{Keywords:} collective motion, criticality, pursuit-evasion, active matter, swarm intelligence, phase transitions
