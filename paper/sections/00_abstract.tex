\begin{abstract}
本文研究二维连续追逃任务中,逃跑者集群“更接近临界”是否带来性能增益。与将无外场相变点直接迁移到任务设置不同,本文采用任务内统计判定:在同一任务中,以
\(
\chi = N_e \cdot \mathrm{Var}_t(P(t))
\)
及其辅助指标(\(\chi_{\text{local}}\)、\(\tau_{P,\mathrm{AR1}}\)、\(\xi_{\text{fluct}}\))定义相对“近临界”区域。论文综合了全部已完成实验批次(含高样本主扫描、压力对比、噪声路线与早期试验批次)。结果表明:在中等追捕压力下(\(v_p/v_e \approx 1.0\sim1.3\)),生存率与 \(\chi\) 通常正相关;在高压力(\(v_p/v_e=1.4\))下该关系显著减弱。噪声路线中,\(w_{\text{align}}=1.0\) 的任务扫描给出强正相关(Pearson 0.824,Spearman 0.965),但 phase 识别峰值并不对应任务最优噪声。结论显示,近临界优势具有明确的场景依赖和外场强度边界。
\end{abstract}

