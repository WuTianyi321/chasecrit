\section{讨论}
\subsection{近临界优势的条件性}
E07/E09/E10/E11/E12 的综合结果支持一个一致结论:近临界优势并非在所有外场强度下成立。中等压力下,\(\chi\) 增大通常伴随更高生存率;高压力下(E09 的 \(v_p/v_e=1.4\))该耦合显著弱化。

\subsection{任务内判定与 phase 判定的分离}
E12 的峰值噪声(1.80)并未在 E10/E11 中转化为任务最优噪声。该现象说明“无外场相变识别”与“有外场任务最优”是两个不同层次的问题,应避免直接等同。

\subsection{关于“低噪声高 \(\chi\)”的解释一致性}
在 E10/E11 中,低噪声区域同时呈现更高 \(\chi\) 与更高生存率。若判定标准限定为任务内统计表现,则该区域可被解释为“更接近任务内近临界区”,并不与本文框架矛盾。本文强调的限制仅在于:该“任务内近临界”不应被误读为“无外场 phase 临界点已迁移为任务最优点”。

\subsection{E09 中高 \(\chi\) 区域离散度的来源}
E09 的 \(\mathrm{safe}\)-\(\chi\) 散点在 \(\chi\approx 4.5\) 附近出现较大纵向离散。补充分层分析显示,该离散主要由不同压力层(\(v_p/v_e=0.9,1.3,1.4\))混合造成:在 \(\chi\in[4.2,4.8]\) 区间内,\(\mathrm{safe}\) 方差中约 96.7\% 来自压力层间差异。相应地,固定压力层后离散明显收敛。

基于原始运行数据的分层 bootstrap 进一步表明,这一现象并非“样本量不足”的主要结果:\(v_p/v_e=0.9\) 与 \(1.3\) 下 \(\mathrm{corr}(\mathrm{safe},\chi)\) 的 95\% 区间保持正值,而 \(v_p/v_e=1.4\) 下相关接近零且区间跨零。这支持“高压力下 \(\chi\) 与任务绩效解耦”的机制解释。
与此同时,单次运行层面的原始散点相关接近零,提示该任务中存在较强轨迹级随机性;参数层结论应依赖多 seed 聚合统计而非单点云线性拟合。

\subsection{跨批次一致性}
从 E01 到 E11,随着样本量上升和指标增强,核心趋势保持一致:中间参数带通常优于极端参数带;高涨落代理对性能的解释力随场景压力变化而变化。

\subsection{局限}
现有追捕策略主要是近邻追逐,尚未系统引入预测拦截与协作分工;此外,障碍与反射边界下的高样本复验仍需补齐。
