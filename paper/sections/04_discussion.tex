\section{Discussion}

\subsection{Conditional Nature of Near-Critical Benefits}

Our results establish that near-critical regimes confer performance advantages in pursuit-evasion, but with important qualifications. The positive correlation between criticality proxies and survival under moderate pursuit pressure ($v_p/v_e \approx 1.0$--$1.3$) supports the intuition that enhanced susceptibility and long-range correlations aid collective response to dynamic threats. However, this relationship is bounded: when pursuit intensity exceeds a threshold (approximately $v_p/v_e \approx 1.4$ in our parameterization), the correlation collapses.

This conditional dependence suggests two distinct operational regimes:
\begin{enumerate}
    \item \textbf{Response-limited regime} (moderate pressure): Pursuers are not fast enough to exploit predictability fully. The swarm benefits from coordinated collective response enabled by near-critical organization.
    \item \textbf{Predictability-limited regime} (high pressure): Fast pursuers can intercept aligned groups effectively. Individual unpredictability becomes more valuable than collective coordination, favoring disordered states.
\end{enumerate}

The transition between these regimes implies that optimal swarm strategies must be \textit{context-dependent}, tuning collective organization to environmental threat levels.

\subsection{Task-Internal vs. Phase-Only Criticality}

The disconnect between phase-only critical points and task-optimal parameters has important implications. In the Vicsek-like phase setting (no pursuers, no goals), susceptibility peaks at intermediate noise ($\eta \approx 1.8$), consistent with known order-disorder transitions~\cite{vicsek1995,chate2008}. However, in the full task, performance peaks at low noise ($\eta \approx 0$--$0.2$), where susceptibility is also highest \textit{within the task} but far from the phase critical point.

This non-transferability suggests that when external forcing is integral to the task, the relevant ``critical'' regime is task-redefined, not simply the intrinsic collective phase transition. This aligns with theoretical work on driven critical phenomena, where external fields can shift or destroy critical points.

\subsection{Comparison with Biological Systems}

Our findings relate to observations of collective animal behavior. Starling flocks exhibit scale-free correlations suggesting criticality~\cite{bialek2012}, but predation studies show that fish schools balance cohesion against predator avoidance~\cite{sumpter2010}. The conditional benefit of near-critical organization we observe may reflect evolutionary pressures: biological collectives may operate near criticality when threats are moderate but switch to disordered strategies under extreme threat.

The intermediate optimal alignment we observe differs from typical biological flocking where alignment is often strong~\cite{couzin2002}. This may reflect differences in threat characteristics: biological predators often target isolated individuals (favoring cohesion), while our pursuers use nearest-neighbor targeting that can exploit aligned groups.

\subsection{Implications for Engineered Swarms}

For autonomous swarm design, our results suggest:
\begin{enumerate}
    \item \textbf{Context-aware tuning:} Swarms should dynamically adjust collective organization based on perceived threat intensity.
    \item \textbf{Predictability awareness:} When adversaries can exploit coordination patterns, criticality benefits are bounded.
    \item \textbf{Task-internal validation:} Criticality should be measured and optimized within the task context rather than inferred from simplified phase diagrams.
\end{enumerate}

These principles motivate ongoing work on self-organized criticality with adaptive parameters.

\subsection{Limitations and Future Directions}

\textbf{Pursuer strategy:} Current pursuers use simple nearest-neighbor targeting. Predictive interception or cooperative encirclement may alter the predictability-coordination trade-off.

\textbf{Environmental complexity:} We focused on periodic boundaries without obstacles. Reflecting boundaries and obstacle fields may modify the criticality-performance relationship.

\textbf{Parameter ranges:} Extensions to larger swarms ($N_e \gg 128$) or different density regimes may reveal additional scaling effects.

\textbf{Finite-size analysis:} Systematic finite-size scaling to confirm critical exponents would strengthen claims about true criticality versus enhanced fluctuations.

Future work should address these limitations while exploring self-organized criticality routes where agents autonomously tune parameters to maintain near-critical operation adapted to local conditions.
