\section{Conclusion}

This study investigated whether near-critical collective states benefit evader swarms in pursuit-evasion tasks. Through extensive parameter sweeps and task-internal criticality analysis, we established three main findings:

First, survival performance is maximized at intermediate alignment strengths ($w_{\text{align}} \approx 0.4$--$0.7$), with the specific optimum shifting toward lower alignment as pursuit pressure increases. This intermediate optimum reflects a trade-off between collective coordination (beneficial for finding and reaching safe zones) and predictability (exploitable by pursuers).

Second, under moderate pursuit pressure ($v_p/v_e \approx 1.0$--$1.3$), survival rates correlate positively with criticality proxies including susceptibility and correlation time ($r \approx 0.38$--$0.62$). This supports the hypothesis that near-critical organization enhances collective response capability in dynamic environments.

Third, this criticality-performance relationship is bounded: under high pressure ($v_p/v_e = 1.4$), the correlation collapses ($r \approx 0$), and optimal strategies shift toward disordered, unpredictable behavior. Additionally, intrinsic phase critical points identified without external forcing do not transfer to task-optimal parameters.

These results demonstrate that near-critical advantages in adversarial tasks are \textit{conditional} rather than universal, depending critically on threat intensity and requiring task-internal validation. The findings inform both biological understanding of collective antipredator behavior and engineering design of adaptive swarm systems capable of context-dependent organization tuning.

\subsection*{Data and Code Availability}

All experimental data and analysis code are available in the project repository. Experiments are fully reproducible via configuration files and command-line tools.
