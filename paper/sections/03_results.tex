\section{Results}

\subsection{Performance Landscape Across Alignment Strength}

Figure~\ref{fig:safe_landscape} shows survival fraction $f_{\text{safe}}$ across alignment strength $w_{\text{align}}$ and pursuit intensity $v_p/v_e$.

\begin{figure}[htbp]
\centering
\begin{subfigure}[b]{0.48\textwidth}
\includegraphics[width=\textwidth]{../doc/results_20260206_walign_task_internal_200seeds/figs/safe_vs_w_align.png}
\caption{Survival fraction vs. $w_{\text{align}}$ for moderate pressures}
\end{subfigure}
\hfill
\begin{subfigure}[b]{0.48\textwidth}
\includegraphics[width=\textwidth]{../doc/results_20260206_walign_task_internal_200seeds/figs/heatmap_safe_mean.png}
\caption{Heatmap of survival fraction}
\end{subfigure}
\caption{Survival performance landscape across alignment strength and pursuit pressure. Error bars show 95\% CI.}
\label{fig:safe_landscape}
\end{figure}

Key observations:
\begin{enumerate}
    \item \textbf{Intermediate optima:} For all pursuit intensities, survival is maximized at intermediate $w_{\text{align}}$ (neither fully ordered nor disordered). Optimal $w_{\text{align}}$ shifts from $\approx 0.7$ at $v_p/v_e = 1.1$ to $\approx 0.4$ at $v_p/v_e = 1.2$.
    \item \textbf{Broad optima:} Performance plateaus are relatively flat near optima (differences between $w_{\text{align}} = 0.4$--$0.7$ typically $< 0.02$ in $f_{\text{safe}}$).
    \item \textbf{Pressure effects:} Higher $v_p/v_e$ reduces overall survival but preserves the qualitative pattern of intermediate optima.
\end{enumerate}

\subsection{Criticality Proxies and Their Trends}

Figure~\ref{fig:criticality_landscape} shows criticality proxies versus $w_{\text{align}}$.

\begin{figure}[htbp]
\centering
\begin{subfigure}[b]{0.48\textwidth}
\includegraphics[width=\textwidth]{../doc/results_20260206_walign_task_internal_200seeds/figs/chi_vs_w_align.png}
\caption{Susceptibility $\chi$}
\end{subfigure}
\hfill
\begin{subfigure}[b]{0.48\textwidth}
\includegraphics[width=\textwidth]{../doc/results_20260206_walign_task_internal_200seeds/figs/tau_P_ar1_vs_w_align.png}
\caption{Correlation time $\tau$}
\end{subfigure}
\caption{Criticality proxies as functions of alignment strength.}
\label{fig:criticality_landscape}
\end{figure}

Susceptibility $\chi$ peaks at intermediate-to-low $w_{\text{align}}$ ($\approx 0.15$--$0.30$), while correlation time $\tau$ peaks at slightly higher values ($\approx 0.3$--$0.5$). Performance optima lie to the right of criticality peaks, indicating that maximal criticality does not coincide with maximal performance.

\subsection{Criticality-Performance Relationship}

Figure~\ref{fig:scatter_chi} and Table~\ref{tab:correlations} show correlations between survival and criticality proxies.

\begin{figure}[htbp]
\centering
\includegraphics[width=0.6\textwidth]{../doc/results_20260206_walign_task_internal_200seeds/figs/scatter_safe_vs_chi.png}
\caption{Survival fraction vs. susceptibility $\chi$ across $w_{\text{align}}$ values. Points represent group means; color indicates pursuit intensity.}
\label{fig:scatter_chi}
\end{figure}

\begin{table}[htbp]
\centering
\caption{Correlations between survival fraction and criticality proxies}
\label{tab:correlations}
\begin{tabular}{cccccc}
\toprule
$v_p/v_e$ & $n$ & $r(\text{safe}, \chi)$ & $r(\text{safe}, \tau)$ & $r(\text{safe}, \xi)$ & $w_{\text{opt}}$ \\
\midrule
0.90  & 120 & 0.272 & 0.272 & 0.236 & 0.55 \\
1.00  & 200 & 0.376 & 0.383 & 0.374 & 0.35 \\
1.05  & 200 & 0.441 & 0.369 & 0.371 & 0.55 \\
1.10  & 200 & 0.467 & 0.447 & 0.525 & 0.70 \\
1.15  & 200 & 0.621 & 0.585 & 0.414 & 0.45 \\
1.20  & 200 & 0.492 & 0.379 & 0.523 & 0.40 \\
1.30  & 120 & 0.378 & 0.408 & 0.320 & 0.65 \\
\textbf{1.40}  & 240 & \textbf{0.024} & 0.126 & -0.322 & 0.25 \\
\bottomrule
\end{tabular}
\end{table}

\textbf{Moderate pressure ($v_p/v_e \approx 1.0$--$1.3$):} Strong positive correlations ($r \approx 0.38$--$0.62$) indicate that nearer-critical states reliably achieve better survival.

\textbf{High pressure ($v_p/v_e = 1.4$):} The correlation collapses ($r = 0.024$ for $\chi$, $r = -0.322$ for $\xi$). Survival becomes uncorrelated with traditional criticality proxies, and optimal strategy shifts to low alignment ($w_{\text{align}} \approx 0.25$).

\subsection{Pressure-Dependent Relationship Collapse}

The transition from positive to negligible correlation is shown in Figure~\ref{fig:pressure_contrast}.

\begin{figure}[htbp]
\centering
\begin{subfigure}[b]{0.48\textwidth}
\includegraphics[width=\textwidth]{../doc/results_20260206_walign_task_internal_200seeds/figs/scatter_safe_vs_chi.png}
\caption{Moderate pressure ($v_p/v_e = 1.1$): $r = 0.467$}
\end{subfigure}
\hfill
\begin{subfigure}[b]{0.48\textwidth}
\includegraphics[width=\textwidth]{../doc/results_20260206_walign_pressure_091314_sr14_240seeds/figs/scatter_safe_vs_chi.png}
\caption{High pressure ($v_p/v_e = 1.4$): $r = 0.024$}
\end{subfigure}
\caption{Criticality-performance relationship under different pursuit pressures.}
\label{fig:pressure_contrast}
\end{figure}

\textbf{E09 variability analysis:} In E09 merged scatter, substantial vertical dispersion exists near $\chi \approx 4.5$. Layered analysis reveals this dispersion stems primarily from between-pressure baseline differences: at $\chi \in [4.2, 4.8]$, approximately 96.7\% of safe fraction variance is explained by pressure layer ($v_p/v_e = 0.9, 1.3, 1.4$). Within fixed pressure layers, dispersion is markedly reduced.

Single-run level correlations are weak (E09 overall: $r \approx -0.016$), indicating strong trajectory-level randomness. Parameter-level conclusions rely on multi-seed aggregate statistics rather than single point-cloud fitting.

\subsection{Noise Route: Internal vs. External Criticality}

Complementary sweeps over angular noise $\eta$ at fixed $w_{\text{align}}$ show strong within-task correlations (Figure~\ref{fig:noise_sweep}).

\begin{figure}[htbp]
\centering
\begin{subfigure}[b]{0.48\textwidth}
\includegraphics[width=\textwidth]{../doc/results_20260206_task_noise_w10_sr11_100seeds/figs/safe_vs_noise.png}
\caption{Survival vs. noise ($w_{\text{align}} = 1.0$)}
\end{subfigure}
\hfill
\begin{subfigure}[b]{0.48\textwidth}
\includegraphics[width=\textwidth]{../doc/results_20260206_task_noise_w10_sr11_100seeds/figs/scatter_safe_vs_chi.png}
\caption{Survival vs. $\chi$ ($w_{\text{align}} = 1.0$)}
\end{subfigure}
\caption{Noise route results. Strong positive correlation ($r = 0.824$) within the task.}
\label{fig:noise_sweep}
\end{figure}

However, comparison with phase-only identification (no pursuers, no safe zones) reveals critical differences (Table~\ref{tab:phase_comparison}).

\begin{table}[htbp]
\centering
\caption{Phase-only vs. task settings}
\label{tab:phase_comparison}
\begin{tabular}{lcc}
\toprule
Setting & Peak $\chi$ at & Peak survival at \\
\midrule
Phase only (no pursuers) & $\eta = 1.8$ & N/A \\
Task ($w_{\text{align}} = 1.0$) & $\eta = 0.0$ & $\eta = 0.2$ \\
Task ($w_{\text{align}} = 0.6$) & $\eta = 0.0$ & $\eta = 0.0$ \\
\bottomrule
\end{tabular}
\end{table}

\textbf{Key finding:} Intrinsic phase critical points do not transfer to task-optimal parameters. Phase-only $\chi$ peaks at $\eta = 1.8$, while task performance peaks at low noise ($\eta \approx 0$--$0.2$), where susceptibility is highest \textit{within} the task but far from the phase critical point.

\subsection{Summary of Findings}

\begin{enumerate}
    \item Intermediate alignment ($w_{\text{align}} \approx 0.4$--$0.7$) maximizes survival across pursuit pressures.
    \item Under moderate pressure ($v_p/v_e \approx 1.0$--$1.3$), survival correlates positively with criticality proxies ($r \approx 0.38$--$0.62$).
    \item Under high pressure ($v_p/v_e = 1.4$), this correlation collapses ($r \approx 0$).
    \item Within-task criticality correlates with performance, but intrinsic phase critical points do not predict task optima.
\end{enumerate}
