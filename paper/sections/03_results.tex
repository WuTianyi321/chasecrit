\section{实验结果(全量批次综合)}
\subsection{批次总览}
为覆盖全部已完成实验,本文将有结构化聚合表(可计算统计量)的批次记为 E 系列,将仅用于烟雾测试或流程验证、未保留完整聚合表的批次记为 P 系列。

\begin{longtable}{p{1.2cm}p{3.4cm}p{1.7cm}p{1.7cm}p{1.9cm}p{3.5cm}}
\caption{E 系列(可量化)实验批次总览与关键统计}\label{tab:all_quant}\\
\toprule
ID & 实验类型 & seeds/格点 & steps & 最优 \(\mathrm{safe}\) & \(\mathrm{safe}\)-\(\chi\) 关系 \\
\midrule
\endfirsthead
\toprule
ID & 实验类型 & seeds/格点 & steps & 最优 \(\mathrm{safe}\) & \(\mathrm{safe}\)-\(\chi\) 关系 \\
\midrule
\endhead
E01 & 固定追捕者扫描(初版) & 10 & 600 & \(sr=1.00,w=0.30\), 0.3055 & 指标版本不含 \(\chi\) \\
E02 & 固定追捕者扫描(扩展) & 15 & 600 & \(sr=1.00,w=0.65\), 0.4214 & 平均相关 0.120 \\
E03 & 固定追捕者扫描(50 seeds) & 50 & 600 & \(sr=1.00,w=0.35\), 0.3966 & 平均相关 0.372 \\
E04 & 固定追捕者扫描(50 seeds, 指标增强) & 50 & 600 & \(sr=1.00,w=0.35\), 0.3966 & 平均相关 0.372 \\
E05 & 固定追捕者扫描(100 seeds, \(sr\in\{1.1,1.2\}\)) & 100 & 600 & \(sr=1.10,w=0.70\), 0.3725 & 平均相关 0.399 \\
E06 & 任务内 \(w_{\text{align}}\) 扫描(中样本) & 80 & 600 & \(sr=1.00,w=0.95\), 0.3900 & 平均相关 0.401 \\
E07 & 任务内 \(w_{\text{align}}\) 扫描(高样本) & 200 & 600 & \(sr=1.00,w=0.35\), 0.3893 & 平均相关 0.479 \\
E08 & 压力对比(0.9/1.3/1.4) & 120 & 600 & \(sr=0.90,w=0.55\), 0.4092 & 平均相关 0.193 \\
E09 & 压力对比(\(sr=1.4\)扩样) & 120/240 & 600 & \(sr=0.90,w=0.55\), 0.4092 & 平均相关 0.225 \\
E10 & 任务噪声扫描(\(w=0.6\)) & 100 & 600 & noise=0.00, 0.3567 & Pearson 0.814 \\
E11 & 任务噪声扫描(\(w=1.0\)) & 100 & 600 & noise=0.20, 0.3387 & Pearson 0.824 \\
E12 & Phase 噪声识别(无追捕/无安全区) & 100 & 1200 & \(\mathrm{safe}\) 不适用 & \(\chi\) 峰值在 noise=1.80 \\
\bottomrule
\end{longtable}

\begin{longtable}{p{1.2cm}p{4.2cm}p{2.0cm}p{6.3cm}}
\caption{P 系列(流程/烟雾)批次及其作用}\label{tab:all_pilot}\\
\toprule
ID & 批次类型 & 是否量化汇总 & 作用 \\
\midrule
\endfirsthead
\toprule
ID & 批次类型 & 是否量化汇总 & 作用 \\
\midrule
\endhead
P01 & Phase smoke & 否 & 验证 phase-sweep 管线与绘图流程可用。 \\
P02 & Phase baseline & 否 & 早期 phase 识别趋势观察。 \\
P03 & Phase wide & 否 & 扩大噪声范围后的早期可视化验证。 \\
P04 & Phase wider & 否(仅图) & 后续高样本 phase 扫描前的参数摸底。 \\
P05 & Task noise smoke & 否 & 验证任务噪声扫描管线与散点图生成。 \\
P06 & Task noise early (\(w=1.0\)) & 否 & 任务噪声路线早期趋势勘探。 \\
\bottomrule
\end{longtable}

\subsection{早期到高样本的演化结果(E01--E06)}
E01--E05 展示了固定追捕者数量扫描从低样本到高样本、从单指标到多指标的演化过程。E01 中尚未纳入 \(\chi\) 指标;E02 开始引入 \(\chi\),并首次在格点均值层面观察到正相关(平均相关 0.120);E03/E04 在 50 seeds 下将相关提升到 0.372;E05 在 100 seeds、\(v_p/v_e\in\{1.1,1.2\}\) 下维持正相关(0.399)。

E06 作为任务内 \(w_{\text{align}}\) 路线的中样本版本(80 seeds),与 E07(200 seeds)在趋势方向上保持一致:\(\mathrm{safe}\) 与 \(\chi\) 的整体耦合为正,但最优 \(w_{\text{align}}\) 呈带状而非尖峰点。

\begin{figure}[htbp]
\centering
\begin{subfigure}[b]{0.48\textwidth}
\includegraphics[width=\textwidth]{../doc/results_20260205_fixedNp_scan_expanded/figs/safe_vs_w_align.png}
\caption{E02: \(\mathrm{safe}\) vs \(w_{\text{align}}\)}
\end{subfigure}
\hfill
\begin{subfigure}[b]{0.48\textwidth}
\includegraphics[width=\textwidth]{../doc/results_20260205_fixedNp_scan_expanded/figs/chi_vs_w_align.png}
\caption{E02: \(\chi\) vs \(w_{\text{align}}\)}
\end{subfigure}
\caption{早期固定追捕者扫描中引入 \(\chi\) 后的趋势}
\label{fig:e02_early}
\end{figure}

\begin{figure}[htbp]
\centering
\begin{subfigure}[b]{0.48\textwidth}
\includegraphics[width=\textwidth]{../doc/results_20260206_fixedNp_sr11_sr12_100seeds/figs/safe_vs_w_align.png}
\caption{E05: \(\mathrm{safe}\) vs \(w_{\text{align}}\)}
\end{subfigure}
\hfill
\begin{subfigure}[b]{0.48\textwidth}
\includegraphics[width=\textwidth]{../doc/results_20260206_fixedNp_sr11_sr12_100seeds/figs/chi_vs_w_align.png}
\caption{E05: \(\chi\) vs \(w_{\text{align}}\)}
\end{subfigure}
\caption{高样本固定追捕者扫描下的主趋势延续}
\label{fig:e05_highsample}
\end{figure}

\subsection{\(w_{\text{align}}\) 主线与压力对比}
E07 与 E09 构成本文主结果骨架。E07(高样本)显示在 \(v_p/v_e=1.0\sim1.2\) 范围内,\(\mathrm{safe}\) 与 \(\chi\) 整体正相关;E09 显示该关系在高压力端(\(v_p/v_e=1.4\))显著减弱。

\begin{figure}[htbp]
\centering
\begin{subfigure}[b]{0.48\textwidth}
\includegraphics[width=\textwidth]{../doc/results_20260206_walign_task_internal_200seeds/figs/safe_vs_w_align.png}
\caption{E07: \(\mathrm{safe}\) vs \(w_{\text{align}}\)}
\end{subfigure}
\hfill
\begin{subfigure}[b]{0.48\textwidth}
\includegraphics[width=\textwidth]{../doc/results_20260206_walign_task_internal_200seeds/figs/chi_vs_w_align.png}
\caption{E07: \(\chi\) vs \(w_{\text{align}}\)}
\end{subfigure}
\caption{高样本任务内 \(w_{\text{align}}\) 扫描结果}
\label{fig:e07_main}
\end{figure}

\begin{figure}[htbp]
\centering
\includegraphics[width=0.68\textwidth]{../doc/results_20260206_walign_pressure_091314_sr14_240seeds/figs/scatter_safe_vs_chi.png}
\caption{E09: 压力对比中的 \(\mathrm{safe}\)-\(\chi\) 关系}
\label{fig:e09_pressure}
\end{figure}

\subsection{E09 离散度诊断}
在 E09 合并散点中,\(\chi\approx4.5\) 区域存在明显纵向离散。按压力分层后可见该离散主要来自 \(v_p/v_e\) 层间基线差异:相同 \(\chi\) 水平下,\(v_p/v_e=0.9\) 的 \(\mathrm{safe}\) 系统性高于 \(1.3\) 与 \(1.4\)。在 \(\chi\in[4.2,4.8]\) 区间,\(\mathrm{safe}\) 方差约 96.7\% 可由层间差异解释,说明该现象主要是条件混合效应而非单一指标噪声。

若不做任何聚合、直接在单次运行层面考察 \((\chi,\mathrm{safe\_frac})\),相关性较弱:E09 全部原始点的 Pearson 相关约为 \(-0.016\),且各压力层分别约为 \(-0.025\)(\(v_p/v_e=0.9\))、\(0.051\)(\(1.3\))与 \(0.018\)(\(1.4\))。因此,本文对“近临界优势”的讨论以参数格点统计量(多 seed 汇总)为主,并将原始点云解释为高随机性背景下的微观离散。

\begin{figure}[htbp]
\centering
\includegraphics[width=0.68\textwidth]{../doc/results_20260206_walign_pressure_091314_sr14_240seeds/figs/scatter_safe_vs_chi_by_sr.png}
\caption{E09: 按压力分层后的 \(\mathrm{safe}\)-\(\chi\) 散点}
\label{fig:e09_by_sr}
\end{figure}

\subsection{噪声路线全量结果(关键对照)}
E10 与 E11 均显示任务噪声扫描中 \(\mathrm{safe}\)-\(\chi\) 强正相关。尤其 E11(\(w_{\text{align}}=1.0\))给出 Pearson 0.824、Spearman 0.965。E12 则表明无外场 phase 设置中 \(\chi\) 峰值位于 noise=1.80,与任务最优噪声不一致。

\begin{table}[htbp]
\centering
\caption{噪声路线关键统计(修正后的峰值位置与数值)}
\label{tab:noise_key}
\begin{tabular}{cccccccc}
\toprule
ID & 设置 & seeds & noise@safe\(_{\max}\) & safe\(_{\max}\) & noise@\(\chi_{\max}\) & \(\chi_{\max}\) & \(\mathrm{corr}(\mathrm{safe},\chi)\) \\
\midrule
E10 & Task, \(w=0.6\) & 100 & 0.00 & 0.3567 & 0.00 & 4.2812 & 0.814 \\
E11 & Task, \(w=1.0\) & 100 & 0.20 & 0.3387 & 0.00 & 4.2378 & 0.824 \\
E12 & Phase & 100 & -- & -- & 1.80 & 13.2085 & -- \\
\bottomrule
\end{tabular}
\end{table}

\begin{figure}[htbp]
\centering
\begin{subfigure}[b]{0.48\textwidth}
\includegraphics[width=\textwidth]{../doc/results_20260206_task_noise_w10_sr11_100seeds/figs/safe_vs_noise.png}
\caption{E11: \(\mathrm{safe}\) vs noise}
\end{subfigure}
\hfill
\begin{subfigure}[b]{0.48\textwidth}
\includegraphics[width=\textwidth]{../doc/results_20260206_task_noise_w10_sr11_100seeds/figs/scatter_safe_vs_chi.png}
\caption{E11: \(\mathrm{safe}\)-\(\chi\) 散点}
\end{subfigure}
\caption{E11(\(w=1.0\))中观察到的强正相关}
\label{fig:e11_noise}
\end{figure}

\begin{figure}[htbp]
\centering
\begin{subfigure}[b]{0.48\textwidth}
\includegraphics[width=\textwidth]{../doc/results_20260206_task_noise_w06_sr11_100seeds/figs/scatter_safe_vs_chi.png}
\caption{E10: Task 噪声扫描对照}
\end{subfigure}
\hfill
\begin{subfigure}[b]{0.48\textwidth}
\includegraphics[width=\textwidth]{../doc/results_20260206_phase_noise_100seeds_steps1200/figs/chi_vs_noise.png}
\caption{E12: Phase 噪声识别}
\end{subfigure}
\caption{任务噪声与 phase 噪声结果的并列比较}
\label{fig:e10_e12}
\end{figure}

\subsection{全量批次图表索引}
为确保正文覆盖全部批次,表~\ref{tab:figure_index} 给出 E/P 系列对应的代表性图表类别。E 系列均对应完整趋势图与散点图;P 系列用于流程与绘图链路验证,其图表仅用于方法学完备性说明,不参与主要定量结论。

\begin{table}[htbp]
\centering
\caption{全量批次代表性图表索引(按实验编号)}
\label{tab:figure_index}
\begin{tabular}{ccc}
\toprule
实验编号 & 主要图表类别 & 在本文中的作用 \\
\midrule
E01--E05 & \(w_{\text{align}}\) 趋势图与热图 & 固定追捕者扫描基线与扩样 \\
E06--E07 & 任务内 \(w_{\text{align}}\) 趋势与散点 & 近临界主线证据(中样本/高样本) \\
E08--E09 & 压力分层趋势与散点 & 外场强度对耦合关系的调制 \\
E10--E11 & 噪声趋势与 \(\mathrm{safe}\)-\(\chi\) 散点 & 任务噪声路线与强正相关检验 \\
E12 & phase 噪声趋势图 & 任务判定与 phase 判定分离 \\
P01--P06 & 简化趋势图(烟雾/流程) & 管线连通性与早期参数摸底 \\
\bottomrule
\end{tabular}
\end{table}
