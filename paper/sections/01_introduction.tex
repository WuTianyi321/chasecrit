\section{引言}
群体系统在临界附近常表现出高易感性、长相关尺度与较强扰动响应能力,这使“临界性是否提升任务性能”成为跨物理与生物系统的重要问题~\cite{vicsek1995,chate2008,mora2011}。在追逃对抗中,该问题进一步复杂化:逃跑者需要在协同与灵活之间平衡,而追捕者可能利用可预测性进行压制。

本文关注逃跑者集群,检验以下命题:
\begin{enumerate}
\item 任务内部更“近临界”的参数区是否对应更高生存率;
\item 该关系是否随追捕压力变化;
\item 无外场 phase 识别得到的“临界点”能否迁移为任务最优点。
\end{enumerate}

为避免跨场景先验引入偏差,本文采用任务内统计判定并对全部实验批次进行统一综合。

