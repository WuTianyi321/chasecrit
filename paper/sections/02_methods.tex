\section{任务与方法}
\subsection{场景、状态变量与边界映射}
环境为二维连续空间 \(\Omega=[0,L_x)\times[0,L_y)\),离散时间步长为 \(\Delta t\)。
逃跑者集合记为 \(\mathcal{E}\)(规模 \(N_e\)),追捕者集合记为 \(\mathcal{P}\)(规模 \(N_p\)),安全区集合记为 \(\mathcal{Z}\)(动态规模,最多 \(K_{\max}\) 个有效区)。
状态变量包括:
\begin{itemize}
\item 逃跑者位置/速度 \((x_i(t),v_i(t))\), \(i\in\mathcal{E}\);
\item 追捕者位置/速度 \((y_m(t),u_m(t))\), \(m\in\mathcal{P}\);
\item 安全区位置/速度 \((z_k(t),q_k(t))\)、容量 \(C_k\)、占用 \(O_k(t)\)、有效标记 \(a_k(t)\)。
\end{itemize}
边界映射记为 \(\mathcal{B}(\cdot)\):
\begin{itemize}
\item 周期边界:\(\mathcal{B}(x)=x\bmod (L_x,L_y)\);
\item 反射边界:按 \(x^\ast=\mathrm{mod}(x,2L)\) 折叠到 \([0,L]\),若发生折返则对应速度分量取反。
\end{itemize}
仿真中每一步先更新安全区,再更新逃跑者与追捕者,最后执行捕获与入区判定。

\subsection{逃跑者运动更新}
每步仅对仍存活且未入区个体(记作 \(\mathcal{A}(t)\))更新。对 \(i\in\mathcal{A}(t)\),定义周期最短位移
\(\Delta_{ij}(t)\)、\(\Delta_{im}^{(p)}(t)\)、\(\Delta_{ik}^{(z)}(t)\)。
单位化算子记为 \(\mathrm{unit}(\cdot)\)。

邻域对齐与排斥项:
\begin{align}
d_i^{\text{align}} &= \mathrm{unit}\!\left(\sum_{j\neq i,\ \|\Delta_{ij}\|^2\le r_{\text{nbr}}^2}\mathrm{unit}(v_j)\right),\\
d_i^{\text{sep}} &= s_{\text{sep}}\cdot \mathrm{unit}\!\left(-\sum_{j\neq i,\ \|\Delta_{ij}\|^2\le r_{\text{sep}}^2}\frac{\Delta_{ij}}{\max(\|\Delta_{ij}\|^2,\varepsilon)}\right).
\end{align}
追捕规避项:
\begin{equation}
d_i^{\text{avoid}}=\mathrm{unit}\!\left(-\sum_{m,\ \|\Delta_{im}^{(p)}\|^2\le r_{\text{pred}}^2}\frac{\Delta_{im}^{(p)}}{\max(\|\Delta_{im}^{(p)}\|^2,\varepsilon)}\right).
\end{equation}
目标项:在可检测有效安全区集合
\(\mathcal{Z}_i(t)=\{k:a_k(t)=1,\ \|\Delta_{ik}^{(z)}\|^2\le r_{\text{detect}}^2\}\)
中选最近区 \(k^\star\)。若 \(\mathcal{Z}_i(t)\neq\varnothing\),
\begin{equation}
d_i^{\text{goal}}=\mathrm{unit}(\Delta_{ik^\star}^{(z)}),\quad I_i^{\text{goal}}=1;
\end{equation}
否则 \(I_i^{\text{goal}}=0\),探索方向为当前航向(若速度近零则随机单位向量)\(d_i^{\text{explore}}\)。

合成方向采用两种控制模式:
\begin{itemize}
\item 传统独立权重模式(legacy):
\begin{equation}
\tilde d_i = w_{\text{align}}d_i^{\text{align}}
+ w_{\text{avoid}}d_i^{\text{avoid}}
+ w_{\text{goal}}I_i^{\text{goal}}d_i^{\text{goal}}
+ w_{\text{explore}}(1-I_i^{\text{goal}})d_i^{\text{explore}}
+ d_i^{\text{sep}}.
\end{equation}
\item 单参数占比模式(share):令 \(\lambda\in[0,1]\) 表示对齐项占比(实现中 \(\lambda=w_{\text{align}}\)),先构造非对齐复合方向
\begin{equation}
d_i^{\text{non}}=\mathrm{unit}\!\left(
w_{\text{avoid}}d_i^{\text{avoid}}
+ w_{\text{goal}}I_i^{\text{goal}}d_i^{\text{goal}}
+ w_{\text{explore}}(1-I_i^{\text{goal}})d_i^{\text{explore}}
+ d_i^{\text{sep}}
\right),
\end{equation}
再按
\begin{equation}
\tilde d_i=\lambda d_i^{\text{align}}+(1-\lambda)d_i^{\text{non}}.
\end{equation}
\end{itemize}
令 \(d_i=\mathrm{unit}(\tilde d_i)\),并施加角噪声
\(\theta_i\sim\mathcal U[-\eta,\eta]\)(\(\eta=\text{angle\_noise}\)):
\begin{equation}
d_i^{\eta}=\mathrm{unit}(R(\theta_i)d_i).
\end{equation}
速度采用惯性混合更新:
\begin{equation}
v_i(t+1)=(1-\alpha)v_i(t)+\alpha v_e d_i^{\eta},\quad \alpha=\text{inertia}.
\end{equation}
位置更新为
\begin{equation}
x_i(t+1)=\mathcal{B}\!\left(x_i(t)+v_i(t+1)\Delta t\right).
\end{equation}

\subsection{追捕者更新、捕获与入区规则}
追捕者速度上限设为 \(v_p=\gamma v_e\)(\(\gamma=v_p/v_e\))。
当前策略为最近邻追逐:
\begin{equation}
j^\star(m)=\arg\min_{j\in\mathcal{A}(t)}\|\Delta_{mj}^{(e)}(t)\|^2,\qquad
u_m(t+1)=v_p\cdot \mathrm{unit}\!\left(\Delta_{mj^\star(m)}^{(e)}(t)\right),
\end{equation}
\begin{equation}
y_m(t+1)=\mathcal{B}\!\left(y_m(t)+u_m(t+1)\Delta t\right).
\end{equation}

捕获规则为瞬时捕获:若存在 \(m\) 使 \(\|x_i-y_m\|\le r_{\text{cap}}\),则逃跑者 \(i\) 在该步记为 captured。

入区规则按安全区逐个处理:对有效区 \(k\),找出 \(\|x_i-z_k\|\le r_{\text{safe}}\) 的候选并随机打乱;按顺序填充容量,若
\(O_k(t)\ge C_k\) 则该区立即失效(逻辑消失)。

\subsection{安全区 G0 缓慢移动与刷新机制}
安全区采用 G0 常速随机游走。每隔 \(T_{\text{turn}}\) 步,方向旋转角
\(\phi\sim\mathcal U[-\phi_{\max},\phi_{\max}]\),随后归一化并乘以常速 \(v_z\);位置更新:
\begin{equation}
z_k(t+1)=\mathcal{B}\!\left(z_k(t)+q_k(t+1)\Delta t\right).
\end{equation}
刷新规则:
\begin{itemize}
\item 若当前有效区数为 0,则强制刷新 1 个安全区;
\item 若有效区数 \(<K_{\max}\),以概率 \(p_{\text{spawn}}\) 触发随机刷新;
\item 反射边界下刷新点从边界采样,周期边界下从域内均匀采样;
\item 刷新点需满足与其他安全区、追捕者(及障碍,如启用)的最小距离约束。
\end{itemize}

\subsection{参数固定项与扫描项(写明 \(w_{\text{align}}\) 扫描时其余参数)}
表~\ref{tab:param_protocol} 给出本文主结论所用实验族的固定参数与扫描参数。除特别注明外,未扫描参数均固定为表中数值。

\begin{table}[htbp]
\centering
\caption{主要实验族的参数协议(固定项与扫描项)}
\label{tab:param_protocol}
\begin{tabular}{p{2.1cm}p{5.8cm}p{5.0cm}}
\toprule
实验族 & 固定参数(核心) & 扫描参数 \\
\midrule
E07(任务内 \(w_{\text{align}}\), 200 seeds) &
\(N_e=128\), \(N_p=2\), \(v_e=1.0\), inertia \(=0.35\), \(r_{\text{nbr}}=6.0\), \(r_{\text{pred}}=10.0\), \(r_{\text{detect}}=4.0\), \(r_{\text{sep}}=1.25\), \(s_{\text{sep}}=1.0\), \(w_{\text{goal}}=1.2\), \(w_{\text{avoid}}=1.5\), \(w_{\text{explore}}=0.4\), angle\_noise \(=0.12\), boundary=periodic, \(K_{\max}=4\), \(r_{\text{safe}}=1.0\), \(v_z=0.1\), steps \(=600\) &
\(w_{\text{align}}\in\{0,0.05,\ldots,1.0\}\),
\(v_p/v_e\in\{1.0,1.05,1.1,1.15,1.2\}\) \\
\midrule
E09(压力对比) &
与 E07 相同(含 angle\_noise \(=0.12\)) &
\(w_{\text{align}}\in\{0,0.05,\ldots,1.0\}\),
\(v_p/v_e\in\{0.9,1.3,1.4\}\) \\
\midrule
E10/E11(任务噪声) &
与 E07 相同,且 \(v_p/v_e=1.1\);E10 固定 \(w_{\text{align}}=0.6\),E11 固定 \(w_{\text{align}}=1.0\) &
angle\_noise \(\in\{0,0.2,\ldots,3.0\}\) \\
\midrule
E12(phase 噪声识别) &
\(N_p=0\), \(K_{\max}=0\), \(w_{\text{goal}}=w_{\text{avoid}}=w_{\text{explore}}=0\), \(w_{\text{align}}=1.0\), 其余动力学参数同上, steps \(=1200\) &
angle\_noise \(\in\{0,0.2,\ldots,3.0\}\) \\
\bottomrule
\end{tabular}
\end{table}

单参数占比路线(share)用于后续实验:固定 angle\_noise \(=0\),扫描 \(\lambda=w_{\text{align}}\in[0,1]\),其余权重 \(w_{\text{goal}},w_{\text{avoid}},w_{\text{explore}}\) 固定,从而避免“多权重同时竞争”带来的解释歧义。

\subsection{任务内临界性判定}
主判定指标为
\begin{equation}
\chi = N_e \cdot \mathrm{Var}_t(P(t)),
\end{equation}
其中 \(P(t)\) 为存活逃跑者速度极化度。辅助指标为
\(\chi_{\text{local}}=N_e\cdot \mathrm{Var}(P_{\text{local}})\)、
\(\tau_{P,\mathrm{AR1}}=\frac{1+\rho_1}{1-\rho_1}\)、
\(\xi_{\text{fluct}}\)(涨落相关长度代理)。
本文将“更近临界”定义为同一任务设置内这些统计量的相对升高,而非外部 phase 峰值的直接迁移。

\subsection{统计协议}
每个参数格点采用多随机种子重复,报告均值与 95\% CI。相关性分析使用格点均值层面的 Pearson(必要时补充 Spearman)系数。
