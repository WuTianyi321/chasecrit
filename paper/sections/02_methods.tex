\section{Methods}

\subsection{Simulation Environment}

\subsubsection{Spatial Setup and State Variables}
Simulations were conducted in a two-dimensional continuous space $\Omega=[0,L_x)\times[0,L_y)$ with discrete time steps $\Delta t$. The environment contains three entity types: evaders $\mathcal{E}$ (the swarm, size $N_e$), pursuers $\mathcal{P}$ (adversaries, size $N_p$), and safe zones $\mathcal{Z}$ (dynamic size, maximum $K_{\max}$ active zones).

State variables include:
\begin{itemize}
\item Evader positions and velocities $(x_i(t), v_i(t))$, $i \in \mathcal{E}$;
\item Pursuer positions and velocities $(y_m(t), u_m(t))$, $m \in \mathcal{P}$;
\item Safe zone positions and velocities $(z_k(t), q_k(t))$, capacity $C_k$, occupancy $O_k(t)$, and active flag $a_k(t)$.
\end{itemize}

Boundary mapping $\mathcal{B}(\cdot)$:
\begin{itemize}
\item Periodic: $\mathcal{B}(x) = x \bmod (L_x, L_y)$;
\item Reflecting: Folded via $x^* = \mathrm{mod}(x, 2L)$ to $[0, L]$, with velocity component reversal on fold.
\end{itemize}

Each simulation step updates safe zones first, then evaders and pursuers, followed by capture and zone entry resolution.

\subsubsection{Evader Motion Update}
Only alive evaders not yet in safe zones ($\mathcal{A}(t)$) are updated. For $i \in \mathcal{A}(t)$, let $\Delta_{ij}(t)$ denote periodic shortest displacement, with analogous definitions for $\Delta_{im}^{(p)}(t)$ (to pursuers) and $\Delta_{ik}^{(z)}(t)$ (to zones). The unitization operator is $\mathrm{unit}(\cdot)$.

\textbf{Neighbor alignment and separation:}
\begin{align}
d_i^{\text{align}} &= \mathrm{unit}\!\left(\sum_{j \neq i, \|\Delta_{ij}\|^2 \leq r_{\text{nbr}}^2} \mathrm{unit}(v_j)\right), \\
d_i^{\text{sep}} &= s_{\text{sep}} \cdot \mathrm{unit}\!\left(-\sum_{j \neq i, \|\Delta_{ij}\|^2 \leq r_{\text{sep}}^2} \frac{\Delta_{ij}}{\max(\|\Delta_{ij}\|^2, \varepsilon)}\right).
\end{align}

\textbf{Pursuer avoidance:}
\begin{equation}
d_i^{\text{avoid}} = \mathrm{unit}\!\left(-\sum_{m, \|\Delta_{im}^{(p)}\|^2 \leq r_{\text{pred}}^2} \frac{\Delta_{im}^{(p)}}{\max(\|\Delta_{im}^{(p)}\|^2, \varepsilon)}\right).
\end{equation}

\textbf{Goal direction:} For detectable active zones $\mathcal{Z}_i(t) = \{k: a_k(t)=1, \|\Delta_{ik}^{(z)}\|^2 \leq r_{\text{detect}}^2\}$, select nearest zone $k^\star$. If $\mathcal{Z}_i(t) \neq \varnothing$:
\begin{equation}
d_i^{\text{goal}} = \mathrm{unit}(\Delta_{ik^\star}^{(z)}), \quad I_i^{\text{goal}} = 1;
\end{equation}
otherwise $I_i^{\text{goal}} = 0$ and exploration direction $d_i^{\text{explore}}$ is current heading (or random if near-zero velocity).

\textbf{Control modes:} Two synthesis modes were implemented:
\begin{itemize}
\item \textit{Legacy mode} (independent weights):
\begin{equation}
\tilde{d}_i = w_{\text{align}} d_i^{\text{align}} + w_{\text{avoid}} d_i^{\text{avoid}} + w_{\text{goal}} I_i^{\text{goal}} d_i^{\text{goal}} + w_{\text{explore}} (1-I_i^{\text{goal}}) d_i^{\text{explore}} + d_i^{\text{sep}}.
\end{equation}

\item \textit{Share mode} (single-parameter proportion): Let $\lambda \in [0,1]$ be alignment proportion (implemented as $\lambda = w_{\text{align}}$). First compute non-alignment composite:
\begin{equation}
d_i^{\text{non}} = \mathrm{unit}\!\left(w_{\text{avoid}} d_i^{\text{avoid}} + w_{\text{goal}} I_i^{\text{goal}} d_i^{\text{goal}} + w_{\text{explore}} (1-I_i^{\text{goal}}) d_i^{\text{explore}} + d_i^{\text{sep}}\right),
\end{equation}
then combine:
\begin{equation}
\tilde{d}_i = \lambda d_i^{\text{align}} + (1-\lambda) d_i^{\text{non}}.
\end{equation}
\end{itemize}

Let $d_i = \mathrm{unit}(\tilde{d}_i)$, apply angular noise $\theta_i \sim \mathcal{U}[-\eta, \eta]$:
\begin{equation}
d_i^{\eta} = \mathrm{unit}(R(\theta_i) d_i).
\end{equation}

Velocity updates with inertia coefficient $\alpha$:
\begin{equation}
v_i(t+1) = (1-\alpha) v_i(t) + \alpha v_e d_i^{\eta}.
\end{equation}

Position updates:
\begin{equation}
x_i(t+1) = \mathcal{B}\!\left(x_i(t) + v_i(t+1) \Delta t\right).
\end{equation}

\subsection{Pursuer Update, Capture, and Zone Entry}

Pursuer speed is $v_p = \gamma v_e$ where $\gamma = v_p/v_e$. Current strategy is nearest-neighbor pursuit:
\begin{equation}
j^\star(m) = \arg\min_{j \in \mathcal{A}(t)} \|\Delta_{mj}^{(e)}(t)\|^2, \quad u_m(t+1) = v_p \cdot \mathrm{unit}\!\left(\Delta_{mj^\star(m)}^{(e)}(t)\right).
\end{equation}

Capture is instantaneous: evader $i$ is captured if $\|x_i - y_m\| \leq r_{\text{cap}}$ for any $m$.

Safe zone entry processes zones sequentially. For active zone $k$, candidates satisfying $\|x_i - z_k\| \leq r_{\text{safe}}$ are randomly shuffled and processed in order until capacity $C_k$ is reached, at which point the zone immediately deactivates.

\subsection{Safe Zone Motion and Refresh}

Safe zones follow G0 constant-speed random walk. Every $T_{\text{turn}}$ steps, direction rotates by $\phi \sim \mathcal{U}[-\phi_{\max}, \phi_{\max}]$, then normalized and scaled to constant speed $v_z$:
\begin{equation}
z_k(t+1) = \mathcal{B}\!\left(z_k(t) + q_k(t+1) \Delta t\right).
\end{equation}

Refresh rules:
\begin{itemize}
\item If no active zones exist, force-spawn one zone;
\item If active zones $< K_{\max}$, spawn probabilistically with probability $p_{\text{spawn}}$;
\item Reflecting boundaries spawn from edges, periodic from uniform domain sampling;
\item Spawn points satisfy minimum distance constraints from other zones, pursuers, and obstacles (if enabled).
\end{itemize}

\subsection{Experimental Parameters}

Table~\ref{tab:param_protocol} summarizes fixed and scanned parameters for main experiment families.

\begin{table}[htbp]
\centering
\caption{Parameter protocols for main experiment families}
\label{tab:param_protocol}
\begin{tabular}{p{2.1cm}p{5.8cm}p{5.0cm}}
\toprule
Family & Fixed parameters & Scanned parameters \\
\midrule
E07 (main) &
$N_e=128$, $N_p=2$, $v_e=1.0$, inertia $=0.35$, $r_{\text{nbr}}=6.0$, $r_{\text{pred}}=10.0$, $r_{\text{detect}}=4.0$, $r_{\text{sep}}=1.25$, $s_{\text{sep}}=1.0$, $w_{\text{goal}}=1.2$, $w_{\text{avoid}}=1.5$, $w_{\text{explore}}=0.4$, angle\_noise $=0.12$, periodic boundary, $K_{\max}=4$, $r_{\text{safe}}=1.0$, $v_z=0.1$, steps $=600$ &
$w_{\text{align}}\in\{0,0.05,\ldots,1.0\}$, $v_p/v_e\in\{1.0,1.05,1.1,1.15,1.2\}$ \\
\midrule
E09 (pressure) & Same as E07 & $w_{\text{align}}\in\{0,0.05,\ldots,1.0\}$, $v_p/v_e\in\{0.9,1.3,1.4\}$ \\
\midrule
E10/E11 (noise) & Same as E07, $v_p/v_e=1.1$; E10: $w_{\text{align}}=0.6$, E11: $w_{\text{align}}=1.0$ & angle\_noise $\in\{0,0.2,\ldots,3.0\}$ \\
\midrule
E12 (phase) & $N_p=0$, $K_{\max}=0$, $w_{\text{goal}}=w_{\text{avoid}}=w_{\text{explore}}=0$, $w_{\text{align}}=1.0$, steps $=1200$ & angle\_noise $\in\{0,0.2,\ldots,3.0\}$ \\
\bottomrule
\end{tabular}
\end{table}

The share mode (subsequent experiments) fixes angle\_noise $=0$ and scans $\lambda = w_{\text{align}} \in [0,1]$, avoiding multi-weight competition ambiguities.

\subsection{Criticality Proxies}

We employ task-internal proxies for ``near-critical'' states:

\subsubsection{Global Susceptibility ($\chi$)}
\begin{equation}
\chi = N_e \cdot \mathrm{Var}_t(P(t)),
\end{equation}
where $P(t)$ is the polarization order parameter of alive evaders, time-averaged per episode.

\subsubsection{Local Susceptibility ($\chi_{\text{local}}$)}
\begin{equation}
\chi_{\text{local}} = N_e \cdot \mathrm{Var}(P_{\text{local}}),
\end{equation}
where $P_{\text{local}}(i) = |\sum_{j \in \mathcal{N}_i} \hat{v}_j| / |\mathcal{N}_i|$ and $\mathcal{N}_i$ includes $i$ and neighbors within $r_{\text{nbr}}$.

\subsubsection{Correlation Time ($\tau$)}
From lag-1 autocorrelation $\rho_1$ of $P(t)$ assuming AR(1) dynamics:
\begin{equation}
\tau = \frac{1 + \rho_1}{1 - \rho_1}.
\end{equation}

\subsubsection{Correlation Length ($\xi$)}
Estimated from spatial decay of velocity direction fluctuations at episode end.

\subsection{Statistical Protocol}

Each parameter combination used $n = 100$--$240$ independent random seeds. Results report means and 95\% confidence intervals (normal approximation). Correlation analyses use Pearson $r$ (with Spearman $\rho$ for robustness) computed on group means across parameter values.
